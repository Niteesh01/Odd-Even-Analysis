
\documentclass{acm_proc_article-sp}

\begin{document}

\title{Project Proposal}
\subtitle{Odd/Even Program Analysis}

\numberofauthors{2} 
\author{
\alignauthor
Niteesh Kumar Akurati\\
       \affaddr{IUB}\\
       \affaddr{F16-IG-3001}\\
       \affaddr{Bloomington,Indiana}\\
       \email{akuratin@umail.iu.edu}
\alignauthor
Parag Juneja\\
       \affaddr{IUB}\\
       \affaddr{F16-IG-3009}\\
       \affaddr{Bloomington,Indiana}\\
       \email{pjuneja@iu.edu}
}

\maketitle

\section{Problem Statement}
Air pollution is caused by several pollutants like PM2.5, PM10, Oxides of Nitrogen, Carbon Monoxide, Carbon Dioxide and several meteorological factors like wind, humidity, temperature etc. We as part of our analysis are considering the impact of the "Odd-Even Program" implemented in New Delhi. The program is concentrated across the vehicles. The prominent pollutants emitted by private cars that impact the environment are the particulate matter and nitrogen oxide. To perform a keen analysis of the impact of the program on the pollution levels in the New Delhi we observed the continuous monitoring data of the pollutants NO2 and PM2.5 available from the CPCB (Central Pollution Control Board), India monitoring stations.

The main aim of the analysis would be to measure the impact of the odd/even program implemented in the capital of India, New Delhi on pollution levels of the city.The Odd-Even Program also known as "Road space rationing" in many parts of the world, aimed to reduce the negative impact generated by urban air pollution through artificially restricting demand (vehicle travel) by rationing the scarce common good road capacity, especially during the peak periods by reducing the number of private vehicles that are allowed to commute on a particular day, where the vehicles with odd numbers as their last number of the registration plate is allowed on an odd day and that with the even number are allowed to commute on an even day.

In addition, trucks were allowed to enter India's capital only after 11 p.m., two hours later than the existing restriction. The driving restriction scheme went into effect as a trial for an initial period of 15 days, from 1 to 15 January 2016. The restriction was in force from 8 a.m. till 8 p.m., and traffic was not restricted on Sundays. The scheme was expected to take more than a million private cars off the 
road every day.\cite{RoadSpaceRationing}

 

The reports in regards to the effect of the program on the pollution in the New Delhi were mixed as the result of program is abstract and cannot be exactly attributed to one factor. The success of the program can only be determined by observing the pollution levels within New Delhi and the areas which are closely located to New Delhi called the NCR (National Capital Region) and have the same pollution levels. The pollution is observed in both New Delhi and NCR region in two phases. In the first phase air pollution levels of both New Delhi and NCR are compared for the month December which is the pre-implementation phase of the odd-even program and in second phase we compared the air pollution levels of both New Delhi and NCR during the month January which is the month of the implementation of the odd-even program. This kind of approach is called the difference-in-difference analysis. Through this analysis, we are hoping to provide some valuable insights in regards to the impact of this program on the air pollution of New Delhi during the implementation phase.  

\section{Objectives}
The main objectives of this analysis are: -
1) Studying the impact of pollutants NO2 and PM2.5 on the pollution and observing the dependencies

2)Compute the averages of pollutants (NO2 and PM2.5) of all the four stations in New Delhi on an hourly basis for both pre-implementation and during the implementation phase and compute the average of these four stations together for getting the mean average concentration for New Delhi

3)Compute the averages of Pollutants (NO2 and PM2.5) for Faridabad
during the pre-implementation and post-implementation phase.


4) Apply the difference-in-difference analysis between New Delhi and Faridabad during each phase and observe the air pollution




\section{Background}
An alarming increase of pollution in the metro cities across the world and the efforts the countries are putting into to bring down the pollution levels through the use of intuitive methods like road space rationing have given us a momentum to look into the data related to this method. "Road space rationing, also known as alternate-day travel, driving restriction, no-drive days,  is a travel demand management strategy aimed to reduce the negative externalities generated by urban air pollution or peak urban travel demand in excess of available supply or road capacity, through artificially restricting demand (vehicle travel) by rationing the scarce common good road capacity, especially during the peak periods or during peak pollution events. This objective is achieved by restricting traffic access into an urban cordon area, city center (CBD), or district based upon the last digits of the license number on pre-established days and during certain periods, usually, the peak hours". Many cities permanently initiated this rule and are confirming that the results are good enough. But world's most polluted metros like "New Delhi - Capital city of India", "Beijing - Capital city of China" have only taken steps recently to mitigate the severe air pollution problems they are currently facing. Having access to the public dataset of New Delhi's 
air quality data we are probing deep using Big Data Analytics to analyze the effectiveness of the road rationing scheme which is recently implemented in New Delhi on a trail basis for a span of 15 days as odd-even scheme. \cite{RoadSpaceRationing}



\section{Data Set References}

i)  http://www.cpcb.gov.in/caaqm/
frmUserAvgReportCriteria.aspx




\bibliographystyle{abbrv}
\bibliography{References}

%\balancecolumns 

\end{document}