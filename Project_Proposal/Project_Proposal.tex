% THIS IS SIGPROC-SP.TEX - VERSION 3.1
% WORKS WITH V3.2SP OF ACM_PROC_ARTICLE-SP.CLS
% APRIL 2009
%
% It is an example file showing how to use the 'acm_proc_article-sp.cls' V3.2SP
% LaTeX2e document class file for Conference Proceedings submissions.
% ----------------------------------------------------------------------------------------------------------------
% This .tex file (and associated .cls V3.2SP) *DOES NOT* produce:
%       1) The Permission Statement
%       2) The Conference (location) Info information
%       3) The Copyright Line with ACM data
%       4) Page numbering
% ---------------------------------------------------------------------------------------------------------------
% It is an example which *does* use the .bib file (from which the .bbl file
% is produced).
% REMEMBER HOWEVER: After having produced the .bbl file,
% and prior to final submission,
% you need to 'insert'  your .bbl file into your source .tex file so as to provide
% ONE 'self-contained' source file.
%
% Questions regarding SIGS should be sent to
% Adrienne Griscti ---> griscti@acm.org
%
% Questions/suggestions regarding the guidelines, .tex and .cls files, etc. to
% Gerald Murray ---> murray@hq.acm.org
%
% For tracking purposes - this is V3.1SP - APRIL 2009

\documentclass{acm_proc_article-sp}

\begin{document}

\title{Project Proposal}
\subtitle{Odd/Even Program Analysis}

\numberofauthors{2} 
\author{
\alignauthor
Niteesh Kumar Akurati\\
       \affaddr{IUB}\\
       \affaddr{F16-IG-3001}\\
       \affaddr{Bloomington,Indiana}\\
       \email{akuratin@umail.iu.edu}
\alignauthor
Parag Juneja\\
       \affaddr{IUB}\\
       \affaddr{F16-IG-3009}\\
       \affaddr{Bloomington,Indiana}\\
       \email{pjuneja@iu.edu}
}

\maketitle

\section{Problem Statement}
The main aim of the analysis would be to measure the impact of the odd/even program implemented in the capital of India, New Delhi on pollution levels of the city.

The Odd-Even Program also known as "Road space rationing" in many parts of the world, aimed to reduce the negative externalities generated by urban air pollution through artificially restricting demand (vehicle travel) by rationing the scarce common good road capacity, especially during the peak periods by reducing the number of private vehicles that are allowed to commute on a particular day, where the vehicles with odd numbers as their last number of the registration plate is allowed on an odd day and that with the even number are allowed to commute on an even day.

In addition, trucks were allowed to enter India's capital only after 11 p.m., two hours later than the existing restriction. The driving restriction scheme went into effect as a trial for an initial period of 15 days, from 1 to 15 January 2016. The restriction was in force from 8 a.m. till 8 p.m., and traffic was not restricted on Sundays. The scheme was expected to take more than a million private cars off the 
road every day.\cite{RoadSpaceRationing}




The key deliverables would be:-

i) To analyze the pollutants like CO2, humidity, Nitrogen monoxide etc during and after the program.

ii) Analyzing information from each air quality analysis center of the city and comparing the data among the stations.

iii) Analyzing the effect of the pollutants on the climatic condition.

iv) Analyzing which factors actually contribute to the pollution.

v) Carefully ruling out the factors which are not feasible for comparison.

vi) Producing analysis of the pollution during different periods of the day.


\section{Background}
An alarming increase of pollution in the metro cities across the world and the efforts the countries are putting into to bring down the pollution levels through the use of intuitive methods like road space rationing have given us a momentum to look into the data related to this method. "Road space rationing, also known as alternate-day travel, driving restriction, no-drive days,  is a travel demand management strategy aimed to reduce the negative externalities generated by urban air pollution or peak urban travel demand in excess of available supply or road capacity, through artificially restricting demand (vehicle travel) by rationing the scarce common good road capacity, especially during the peak periods or during peak pollution events. This objective is achieved by restricting traffic access into an urban cordon area, city center (CBD), or district based upon the last digits of the license number on pre-established days and during certain periods, usually, the peak hours". Many cities permanently initiated this rule and are confirming that the results are good enough. But world's most polluted metros like "New Delhi - Capital city of India", "Beijing - Capital city of China" have only taken steps recently to mitigate the severe air pollution problems they are currently facing. Having access to the public dataset of New Delhi's 
air quality data we are probing deep using Big Data Analytics to analyze the effectiveness of the road rationing scheme which is recently implemented in New Delhi on a trail basis for a span of 15 days as odd-even scheme. \cite{RoadSpaceRationing}

\section{Artifacts to be submitted}
The following artifacts are planned to be produced during the course of the project:-

i) Formal project proposal document

ii) Requirement specification document

iii) Technical requirements specification document

iv) Code files

v) Implementation document

vi) Final project analysis document

This is a tentative list of artifacts that are planned to be submitted will add/modify the list during the course of the project


\section{Team}

1) Niteesh Kumar Akurati, HID: F16-IG-3001, Gitlab Username: Niteesh01

2) Parag Juneja, HID: F16-IG-3009, Gitlab Username: paragjuneja

\section{Timeline}
The milestones that we set for project are as follows:-

i) Initial formal project proposal - October 3, 2016

ii) Requirement Specification      - October 13, 2016

iii) Analysis                      - October 23, 2016

iv) Design                         - November 3, 2016

v) Implementation                   - November 15, 2016

vi) Integration                     - November 20, 2016

vii) Testing                        - November 27, 2016

viii) Delivery                      - November 30, 2016

All these milestones are tentative as of now will provide more accurate/ specific dates during the course of the project


\section{Data Set References}

i)  http://www.cpcb.gov.in/caaqm/frmUserAvgReportCriteria.aspx
ii) http://164.100.160.234:9000/


\bibliographystyle{abbrv}
\bibliography{sample}

%\balancecolumns 

\end{document}
